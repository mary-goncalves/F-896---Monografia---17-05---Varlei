\chapter*{Resumo}
\markboth{Resumo}{Resumo}
\addcontentsline{toc}{chapter}{Resumo}

\textit{Clusters}, são agregados de átomos ou moléculas que variam desde três a vários milhares de átomos e possuem propriedades interessantes que os tornam diferentes quando comparados ao material massivo. Essas propriedades dependem fortemente da razão entre o número de átomos presentes na superfície e o número de átomos no volume e também da formação de níveis de energia discretos para os elétrons.  
Com o objetivo de melhor entender essas propriedades, faz se necessário o estudo e compreensão da estabilidade dessas estruturas. Em um artigo seminal publicado por Knight \cite{electronic_Shell_sodium}, medidas da distribuição de massas de \textit{clusters} de sódio, mostravam picos bem definidos e visivelmente maiores que os demais, para \textit{clusters} com $2, 18, 20, 34, 40, 58,92 ...$ átomos, indicando uma maior abundância e estabilidade desses tamanhos de \textit{clusters} em relação aos demais. Eles foram apelidados de "\textit{clusters} com números
\textit{mágicos} de átomos", cujos padrões foram atribuídos aos efeitos de preenchimento das camadas eletrônicas \cite{Brack}. Com intuito de observar e entender melhor esses padrões para os \textit{clusters} de prata, foram realizados experimentos, com incidência de luz ultravioleta e sem a iluminação, para que fosse possível comparar os espectros  de abundância obtidos. Com isso foi possível estudar as tendências da abundância da formação dos \textit{clusters} de prata de diversos tamanhos e correlacionar com sua propriedades eletrônicas e estruturais. Os picos mais energeticamente estáveis foram indicados pelos picos iguais à $ 3,9,21,35...$ átomos, devido ao caráter iônico das partículas produzidas no experimento. Além disso, os números mágicos também poderiam decorrer do potencial de ionização dos agregados, porém isso não foi observado. Outro fato que tornou-se evidente com a realização dos experimentos foi o aumento da na intensidade dos picos quando o LED era usado.

