\chapter*{Resumo}
\markboth{Resumo}{Resumo}
\addcontentsline{toc}{chapter}{Resumo}

\textit{Clusters}, são agregados de átomos ou moléculas que variam desde três a vários milhares átomos e possuem propriedades interessantes que os tornam diferentes quando comparados ao material massivo. Essas propriedades dependem fortemente do tamanho e das estruturas geométricas que os compõem. Isso decorre dos efeitos entre a razão do número de átomos presentes na superfície e o número de átomos no
volume, ademas dos níveis  de energia discretos. Com o objetivo de melhor entender essas propriedades, faz se necessário o estudo e compreensão da estabilidade dessas estruturas. Em um artigo publicado por Knight \cite{electronic_Shell_sodium}, medidas da distribuição de massas de \textit{clusters} de sódio ($Na$), mostram picos bem definidos e visivelmente maiores que os demais, para \textit{clusters} com $2, 18, 20, 34, 40, 58,92 ...$ átomos. Esses picos dizem respeito à maior abundância e estabilidade dos \textit{clusters} atômicos desses tamanhos em relação aos demais. Estes foram apelidados de "\textit{clusters} com números
\textit{mágicos} de átomos", cujos padrões foram atribuídos aos efeitos de preenchimento das camadas eletrônicas \cite{Brack}. Com intuito de observar e entender melhor esses padrões para os \textit{clusters} de prata, foram realizados experimentos, com incidência de luz ultravioleta e sem a iluminação, para que fosse possível comparar os espectros  de abundância obtidos. Com isso foi possível estudar as tendências da abundância da formação dos \textit{clusters} de prata de diversos tamanhos e correlacionar com sua propriedades estruturais. Os picos mais energeticamente estáveis foram indicados pelos picos iguais à $n= 3,9,21,35...$ átomos, devido ao caráter iônico das partículas produzidas no experimento. Além disso, os números mágicos também poderiam decorrer do potencial de ionização dos agregados, porém isso não foi observado. Outro fato que tornou-se evidente com a realização dos experimentos foi a mudança na intensidade dos picos quando o LED encontra-se ligado e quando o LED encontra-se desligado; no primeiro caso   temos uma intensidade dos picos muito maior.

\newpage

$ $