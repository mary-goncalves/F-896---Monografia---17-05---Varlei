\chapter*{Resumo}
\markboth{Resumo}{Resumo}
\addcontentsline{toc}{chapter}{Resumo}

\textit{Clusters}, são agregados de átomos ou moléculas que variam desde três a vários milhares átomos, possuem propriedades interessantes que os tornam diferentes quando comparados ao material massivo. Essas propriedades dependem fortemente do tamanho e das estruturas geométricas que os compõem. Isso decorre dos efeitos entre a razão do número de átomos presentes na superfície e o número de átomos no
volume, ademas dos níveis  de energia discretos. Com o objetivo de melhor entender essas propriedades, faz se necessário o estudo e compreensão da estabilidade dessas estruturas. Em um artigo seminal publicado por Knight \textit{et. al.} \cite{electronic_Shell_sodium}, medidas da distribuição de massas de \textit{clusters} de sódio ($Na$), produzidos em fase gasosa, mostram picos bem definidos e visivelmente maiores que os demais, para \textit{clusters} com $2, 18, 20, 34, 40, 58,92 ...$ átomos. Esses picos dizem respeito à maior abundância e estabilidade dos \textit{clusters} atômicos desses tamanhos em relação aos demais. Apelidou-se a esses \textit{clusters} de "\textit{clusters} com números
\textit{mágicos} de átomos", cujos padrões foram atribuídos aos efeitos de preenchimento das camadas eletrônicas \cite{Brack}. Com intuito de observar e entender melhor esses padrões para os \textit{clusters} de prata, foram realizados experimentos, utilizado uma fonte para a produção
de agregados que possibilita controlar o número de átomos que compõem as nanopartículas, onde foi possível estudar as tendências da abundância da formação dos \textit{clusters} de prata de diversos tamanhos e correlacionar com sua propriedades estruturais. Os picos mais energeticamente estáveis foram indicados pelos picos iguais à $n= 3,9,21,35...$ átomos, devido ao caráter iônico das partículas produzidas no experimentos. Além disso, os números mágicos também ocorrem no potencial de ionização dos agregados. Assim, pretendemos realizar experimentos com incidência de luz ultravioleta (UV), induzindo a sua ionização, e comparar os espectros de abundância obtidos com e sem o uso da luz UV. 


\newpage

$ $