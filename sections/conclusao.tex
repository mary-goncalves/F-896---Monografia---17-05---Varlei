\chapter{Conclusão}
\label{sec:conclusao}

Dado o contexto de enfoque das pesquisa atuais, em que grande parte se volta para o desenvolvimento de nanotecnologia, torna-se de extrema importância conhecer as propriedades em nanoescala dos materiais de interesse.

O objetivo deste trabalho era o estudo das tendências de abundância na formação dos \textit{clusters} de diversos tamanhos e correlacionar com sua propriedades estruturais e energéticas; fato este que já esta bem estabelecido na literatura estrangeira e que está intrinsecamente ligado com a ciência de materiais e física do estado sólido, colaborando fortemente com meu o entendimento das propriedades macroscópicas dos materiais e o comportamento quântico da matéria.

O desenvolvimento do projeto passou por três etapas. Primeiramente foi necessário aprender a utilizar a fonte de \textit{clusters} e durante esse processo foi possível desenvolver habilidades para trabalhar com equipamentos de laboratório que utilizam alto vácuo. Também fui introduzida aos modelos atuais de tratamento de dados, realizando o desenvolvendo rotinas em Python que agilizassem o processo de aquisição e fosse possível tratar os dados brutos adquiridos.

Estabelecida uma certa familiaridade com o equipamento, foi dado início ao planejamento de o sistema de iluminação ultravioleta. Sistema esse que induz a ionização das nanopartículas. Para sua construção foi necessário a montagem e caracterização do aparato.

Com o sistema de iluminação pronto foi possível comparar os espectros de abundância obtidos com e sem o uso da luz UV. Estes resultados que nos permitiu avaliar que os \textit{"números mágicos"} também ocorrem no potencial de ionização dos agregados, porém neste caso, não são exclusivamente devido a este fato. Foi possível notar também, um aumento significativo na intensidade da corrente medida durante os experimentos com a incidência de luz ultravioleta; evidência que contribuí para um maior número na produção de nanopartículas, auxiliando na eficiência da máquina dentro do grupo.

Durante todo o processo tive a oportunidade de desenvolver diversos conhecimentos na área de nanopartículas que culminaram neste trabalho.


