\chapter{Introdução}
\pagenumbering{arabic}

A partir do fim da década de 70, os estudos de estruturas nanométricas começaram a ganhar destaque por sua relevância para a área da biomedicina. Duas décadas depois, o interesse em nanossistemas também ganhou força na área da física e desde então, os estudos nessa área cresceram em ritmo acelerado. As nanoestruturas apresentam propriedades novas e interessantes que as diferem enormemente dos sistemas macroscópicos e possuem com grande potencial tecnológico em áreas como: química \cite{catalise}, eletrônica \cite{semicondutores} e biomedicina \cite{antimicrobial_effects, drug_delivery}.


Neste contexto, os nanoagregados (ou \textit{clusters} atômicos), são um caso interessante, pois podem ser compostos de três a vários milhares de átomos, podendo ser formados por um ou mais elementos, e encontram-se na fronteira entre entre a física atômica e a física da matéria condensada.
%os átomos e o \textit{bulk}\footnote{\textit{Entende-se como "bulk" um conjunto de partículas sólidas grande o suficiente para que a média estatística de suas propriedades seja independente do número de partículas\cite{bulk}}}.
Seu estudo possibilita uma melhor compreensão de como as propriedades macroscópicas surgem do comportamento quântico da matéria \cite{Heer,Brack}.

As propriedades físico-químicas das nanopartículas podem variar de forma abrupta com seu tamanho, principalmente quando se trata de \textit{clusters} atômicos, cujo diâmetro varia até \mbox{1 nm}. Esse fato implica que é possível controlar essas propriedades se seu processo de formação for precisamente controlado  \cite{energetic_thermodynamic}, o que fomenta ainda mais o potencial tecnológico dessas estruturas.


Assim, uma das vertentes dos estudos realizados pelo Grupo   de   Física   de   Nanossistemas   e Materiais  Nanoestruturados  (GFNMN)  do  Departamento de  Física  Aplicada  (DFA),  são nanopartículas metálicas produzidas por uma Fonte de \textit{Clusters} e Agregados (FoCA). Este instrumento produz nano-partículas por um método físico, com controle de seu tamanho e de sua dispersão, além da sua composição. A FoCA foi desenvolvida por Artur Domingues Tavares de Sá e Giulia Di Domenicantonio \cite{tese_artur}, ex-membros do grupo, para possibilitar o estudo mais aprofundado das nanoestruturas.

Para obter a análise de distribuição de tamanho dos \textit{clusters}, por meio da distribuição de massa dessas partículas, é utilizado a técnica de espectrometria de massa por tempo de voo, possibilitando, assim, um estudo sobre suas propriedades em função do tamanho das partículas, o que torna a utilização dessa máquina muito interessante para o grupo.

% Esse tipo de análise permite que seja estudado critérios fundamentais para as tendências estruturais e energéticas.

Em um artigo seminal publicado por Knight \textit{et. al.} \cite{electronic_Shell_sodium}, medidas da distribuição de massas de \textit{clusters} de sódio, produzidos em fase gasosa, mostram picos bem definidos e visivelmente maiores que os demais, para \textit{clusters} com $2, 18, 20, 34, 40, 58,92 ...$ átomos. Esses picos dizem respeito à maior abundância e estabilidade dos \textit{clusters} atômicos desses tamanhos em relação aos demais. Apelidou-se a esses \textit{clusters} de "\textit{clusters} com números
\textit{mágicos} de átomos", cujos padrões foram atribuídos aos efeitos de preenchimento das camadas eletrônicas \cite{Brack}. O modelo quântico de Jellium \cite{jellium} obteve sucesso para explicar os números \textit{mágicos}, mas também existem casos em que \textit{clusters} com número \textit{mágicos} átomos também aparecem devido ao preenchimento de camadas geométricas ou poliédricas, deixando de ser uma propriedade eletrônica.


Os números mágicos não aparecem somente para o elemento sódio, mas sim para uma série de elementos incluindo os metais de transição como \cite{magic_1B}  cobre, prata, ouro, platina, dentre muitos outros.

Um dos objetivos desse trabalho é utilizar a Fonte de \textit{Clusters} e Agregados e a técnica de espectrometria de massa por tempo de voo para estudar as tendências da abundância da formação dos \textit{clusters} de diversos tamanhos e correlacionar com sua propriedades estruturais e energéticas. Pretende-se realizar experimentos com um metal de transição - no caso prata - para verificar o aparecimento dos \textit{clusters} com números \textit{mágicos}. Além disso, os números mágicos também ocorrem no potencial de ionização dos agregados. Assim, pretendemos realizar experimentos com incidência de luz ultravioleta (UV), induzindo a sua ionização, e comparar os espectros de abundância obtidos com e sem o uso da luz UV. 

No capítulo 2 deste trabalho, serão  abordados os embasamentos teóricos sobre o aparecimento dos \textit{clusters} com números \textit{mágicos} sustentado pela literatura disponível. Subsequentemente, no capítulo 3 apresentaremos a máquina utilizada para a produção de \textit{clusters} e as modificações feitas na máquina para conseguir realizar os experimentos com luz UV. Discutiremos no capítulo seguinte as caracterizações
realizadas da prata e os resultados obtidos. Por fim, realizaremos um
compêndio geral do trabalho apresentado, mostrando as principais conclusões. 



 