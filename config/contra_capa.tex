\pagestyle{plain}
%\pagenumbering{arabic}
%\pagenumbering{roman}
%\thispagestyle{empty}
\vspace{-5cm}
\includegraphics[width=.94\textwidth, height=1in,
keepaspectratio=true]{logos/logo_Unicamp}

\begin{center}
{\Large {\sc Universidade Estadual de Campinas \\}

Instituto de Física  Gleb Wataghin \\
\vspace{0.5cm}
}
{\Large {\Large Maria Helena Gonçalves}\\
}

\vspace{3.5cm}
{\huge Números mágicos presentes
\vspace{0.4cm}
nos espectros de massa das nanopartículas de prata}

\vspace{2.5cm}

\end{center}

\begin{flushright}
  \begin{minipage}[c]{.5\textwidth}
        Monografia, apresentada ao Curso de Licenciatura em Física da Universidade Estadual de Campinas como requisito para obtenção do título de licenciatura em Física.
        
        \vspace{.2cm}
        \textbf{Orientador:} Prof. Dr. Varlei Rodrigues
    
  \end{minipage}
\end{flushright}

\begin{center}



\vspace{1cm}

%\noindent {\Large{Orientador:} Prof. Dr. Pedro Orientador} \\


%\vspace{1cm}


%\begin{tabular}{l}

%Departamento de Física Aplicada \\
%Instituto de Física {\em Gleb Wataghin}\\
%Universidade Estadual de Campinas
%\end{tabular}

\end{center}

\vspace{1cm}

\begin{center}

\noindent {\Large{Campinas - SP}} \\
\noindent {\Large{2018}} \\

\end{center}


